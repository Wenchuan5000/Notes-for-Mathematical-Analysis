\usepackage[toc,page]{appendix}


\usepackage[utf8]{inputenc}
\usepackage{amsfonts}
\usepackage{amsthm} %proof
\usepackage{amssymb} %subsetneq
\usepackage{amsmath} %aligned
\usepackage{enumerate}
\usepackage{mathrsfs} %for `\mathscr`

\usepackage{units}

\usepackage{graphicx}
\graphicspath{ {./images/} }

\usepackage{layouts}
%textwidth in cm: \printinunitsof{pt}\prntlen{\textwidth}


\newtheoremstyle{wenchuan}
	{1.25em} % Space above
	{1.25em} % Space below
	{} % Body font
	{} % Indent amount
	{\bfseries} % Theorem head font
	{.} % Punctuation after theorem head
	{0.5em} % Space after theorem head
	{} % Theorem head spec (can be left empty, meaning `normal')


%\theoremstyle{definition}

\theoremstyle{wenchuan}
\newtheorem{definition}{Definition}[section]
\newtheorem{proposition}{Proposition}[section]
\newtheorem{observation}{Observation}[section]
\newtheorem{problem}{Problem}[section]

\newtheorem{theorem}{Theorem}[section]
\newtheorem{corollary}{Corollary}[section]
\newtheorem{lemma}{Lemma}[section]

\newtheorem{example}{Example}[section]
\newtheorem{note}{Note}[section]


%\renewenvironment{proof}{{\bfseries Proof.}}{}


\newcommand{\qedlm}{\hfill \ensuremath{\Box}}
\renewcommand{\qed}{\hfill \ensuremath{\blacksquare}}

\renewcommand{\baselinestretch}{1.25}


% Colors
%================================================
%::::::::::::::::::::::::::::::::::::::::::::::::

\usepackage{hyperref}

%------------------------------------------------

\hypersetup{
    colorlinks=true,
    linkcolor=black,
    filecolor=black,      
    urlcolor=black,
}

%------------------------------------------------

%::::::::::::::::::::::::::::::::::::::::::::::::
%================================================



% Colors
%================================================
%::::::::::::::::::::::::::::::::::::::::::::::::

\usepackage{xcolor}

%------------------------------------------------

\definecolor{red}{HTML}{DF384F}
\definecolor{blue}{HTML}{0940DE}

%------------------------------------------------

%::::::::::::::::::::::::::::::::::::::::::::::::
%================================================



% Title Formatting
%================================================
%::::::::::::::::::::::::::::::::::::::::::::::::

\usepackage{titlesec}

%------------------------------------------------
\titleformat
{\chapter} % command
[display] % shape
{\bfseries\Large\itshape} % format
{\centering Chapter \ \thechapter.} % label
{0ex} % sep
{
%    \rule{\textwidth}{1pt}\vspace{1ex}
%    \vspace{1ex}
	\huge
    \centering
} % before-code
[
%\vspace{-0.3ex}%
%\rule{\textwidth}{0.3pt}
] % after-code
%------------------------------------------------

\titleformat
{\section}
[display]
{\bfseries\large}
{} % label
{0ex} % sep
{
	\S \thesection \ \
} % before-code
[
\vspace{-0.5ex}%
\rule{\textwidth}{0.3pt}\vspace{2ex}
] % after-code

%::::::::::::::::::::::::::::::::::::::::::::::::
%================================================



% Footnote
%================================================
%::::::::::::::::::::::::::::::::::::::::::::::::
\renewcommand{\footnoterule}{
  \kern 2ex
  \hrule width 160pt height 0.3pt
  \kern 2ex
}
%::::::::::::::::::::::::::::::::::::::::::::::::
%================================================