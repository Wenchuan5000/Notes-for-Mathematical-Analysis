\chapter{Metric Spaces and Normed Linear Spaces}
%============================================



\section{Metric Spaces}
%--------------------------------------------

\begin{definition}
	\label{def: metric space}
	Let $X$ be any non-empty set. A \textit{metric on $X$} is a mapping $d: X  \times X \to \mathbb R_{\ge 0}$ which satisfies the \textit{Metric Axioms}. That is, for any $x, y, z \in X$,
	\begin{enumerate}
		\item
		\label{def: metric axioms: identity of indiscernibles}
		$d(x,y) = 0$ iff $x = y$;
		
		\item
		\label{def: metric axioms: symmetry}
		$d(x,y) = d(y, x)$;
		
		\item
		\label{def: metric axioms: triangle inequality}
		$d(x,y) \le d(x, z) + d(z, y)$.
	\end{enumerate}
	
	The ordered pair $(X, d)$ is called a \textit{metric space}.
\end{definition}


\begin{example}
	\label{eg: lebesgue metric}
	
	Given $p \in \overline{\mathbb R}_{> 0}$ (we denote $\overline{\mathbb R}$ for $\mathbb R \cup \{ \pm \infty \}$), let $d_p: \mathbb R^n \times \mathbb R^n \to \mathbb R_{>0}$ ($n \in \mathbb N_{> 0}$) be defined as
	$$
	d_p(\mathbf x, \mathbf y) := \left( \sum_{i = 1}^n | x_i - y_i|^p \right)^\frac{1}{p},
	$$
	then $d_p$ is a \textit{$L^p$ metric} (named after Henri Lebesgue) on $X$.
	
	The \textit{Euclidean metric} on $\mathbb R^n$ is the case as $k = 2$, i.e., $d_p$ is a $L^2$ metric on $X$.
	
	In the case of $L^\infty$, we have
	$$
	d_{\infty} = \max_{i \in \{1, \ldots, n\}} |x_i - y_i|.
	$$
\end{example}


\begin{example}
	\label{eg: discrete metric}
	Let $\mathbb X = (X, d)$ be a metric space, where $X$ is a non-empty set, and the metric $d: X \times X \to \mathbb R_{> 0}$ is defined as
	$$
	d(x,y) :=
	\begin{cases}
		0, & \text{if $x = y$}; \\
		1, & \text{else}.
	\end{cases}
	$$
	
	$d$ is called a \textit{discrete metric} on $X$.
\end{example}


% todo: Q: do i need more eg here?


\section{Open Sets}
%--------------------------------------------


\begin{definition}
	\label{def: open balls}
	
	Let $\mathbb X = (X, d)$ be a metric space
	
	Given $x \in X$ and $\varepsilon \in \mathbb R_{> 0}$.
	
	An \textit{$\varepsilon$-ball} of $x$, denoted by $B(x, \varepsilon)$, is a set defined as
	$$
	B(x, \varepsilon) := \left\{ y \in X : d(x,y) < \varepsilon \right\}.
	$$
\end{definition}


\begin{note}
	\label{note: open balls: notation}
	If there are many metric spaces $\mathbb X = (X, d_\mathbb X)$, $\mathbb Y = (Y, d_\mathbb Y)$, $\mathbb M = (M, d_\mathbb X)$, $\ldots,$ we use the subscript
	$$
	B_{\mathbb X}(x , \varepsilon) \text{ or } B_{d_\mathbb X} (x, \varepsilon)
	$$
	to mention that in which space or under which metric an open ball is defined.
\end{note}


\begin{definition}
	\label{def: open sets}
	A set $A \subseteq X$ is said to be \textit{open in $\mathbb X$} iff for any $x \in A$, there exists $\varepsilon \in \mathbb R^{> 0}$, such that
	$$
	B(x, \varepsilon) \subseteq A.
	$$
\end{definition}


\begin{lemma}
	\label{lm: open balls are open} For any $x \in X$ and any $\varepsilon \in \mathbb R_{>0}$, $B(x, \varepsilon)$ is open in $\mathbb X$.
	
	\begin{proof}
		For any $y \in B(x, \varepsilon)$, by Definition \ref{def: open balls}, $d(x,y) < \varepsilon$.
		
		Let $r = \varepsilon - d(x,y)$. For any $z \in B(y, r)$, $d(y, z) < r$. Then, by Open Axiom \ref{def: metric axioms: triangle inequality},
		$$
		d(x,z) \le d(x, y) + d(y,z) \le \varepsilon - r + r = \varepsilon.
		$$
		Thus, $z \in B(x, \varepsilon)$. As $z$ is arbitrarily picked from $B(y, r)$, we have $B(y, r) \subseteq B(x, \varepsilon)$. By Definition \ref{def: open sets}, $B(x, \varepsilon)$ is open in $\mathbb X$.
	\end{proof}
\end{lemma}



\begin{theorem}
	\label{thm: open set axioms}
	Let $\mathcal T \subseteq 2^X$ be the set of all open subsets of $X$, then $\mathcal T$ satisfies the \textit{Open Set Axioms}. That is,
	\begin{enumerate}
		\item
		\label{thm: open set axioms: necessary elements}
		$\emptyset , X \in \mathcal T$;
		
		\item
		\label{thm: open set axioms: closed under arbitrary intersection}
		for any $\mathcal U \subseteq \mathcal T$, $\bigcup \mathcal U \in \mathcal T$;
		
		\item
		\label{thm: open set axioms: closed under finite intersection}
		for any finite $\mathcal F \subseteq \mathcal T$, $\bigcap \mathcal F \in \mathcal T$.
	\end{enumerate}
	
	\begin{proof} \
		\begin{enumerate}
			\item
			 For any $x \in X$ and any $\varepsilon \in \mathbb R_{> 0}$, by Definition \ref{def: open balls}, $B(x, \varepsilon) \subseteq X$. Thus $X \in \mathcal T$. It is vacuously true that $\emptyset \in \mathcal T$.
			 \qedlm
			 
			 \item
			 Let $\mathcal U \subseteq \mathcal T$. For any $x \in \bigcup \mathcal U$, there exists $U \in \mathcal U$ such that $x \in U$. As $U \in \mathcal T$, there there exists $\varepsilon \in \mathbb R_{> 0}$, such that $B(x, \varepsilon) \subseteq U$. As $U \subseteq \bigcup \mathcal U$, $B \subseteq \bigcup \mathcal U$.
			 \qedlm
			 
			 \item
			 Let $A, B \in \mathcal T$, and let $x \in A \cap B$. As $A \in \mathcal T$, there exists $\varepsilon_A \in \mathbb R_{> 0}$ such that $B(x, \varepsilon_A ) \subseteq A$; similarly, there exists $\varepsilon_B \in \mathbb R_{> 0}$ such that $B(x, \varepsilon_B) \subseteq B$.
			 
			 Now, suppose $\varepsilon_A \le \varepsilon_B$. For any $y \in B(x, \varepsilon_A)$, as
			 $$
			 d(x,y) < \varepsilon_B \le \varepsilon_B,
			 $$
			 by Definition \ref{def: open balls}, $y \in B(x, \varepsilon_A) \subseteq B(x, \varepsilon_B)$. Now, we have $y \in A \cap B$.
			 
			 As $y \in B(x, \varepsilon_A)$ is arbitrarily given, $A \cap B \in \mathcal T$.
			 
			 By deduction, for any finite $\mathcal F \subseteq \mathcal T$, $\bigcap \mathcal F \in \mathcal T$.
		\end{enumerate}
	\end{proof}
\end{theorem}


\begin{theorem}
	\label{thm: topology might not close under infinite intersection}
	$\mathcal T$ is not necessarily closed under infinite intersection.
	
	\begin{proof}
		Seeking a counter-example, let $(\mathbb R, d)$ be a Euclidean metric space. Let
		$$
		\mathcal U = \left\{ \left( -\frac{1}{n}, \frac{1}{n} \right) : n \in \mathbb N_{>0} \right\}.
		$$
		
		For any $U \in \mathcal U$, $U \in \mathcal T$. But for any $n \in \mathbb N_{> 0}$, there exists $N \in \mathbb N_{> n}$, such that
		$$
		\left( -\frac{1}{N}, \frac{1}{N} \right) \subsetneq \left( -\frac{1}{n}, \frac{1}{n} \right).
		$$
		Thus, for any $\bigcap \mathcal U = \{0\}$. By Definition \ref{def: open sets}, $\{0\}$ is not open in $\mathbb X$.
	\end{proof}
\end{theorem}



\begin{corollary}
	\label{cor: open sets: only if union of open}
	A subset $A \subseteq X$ is open only if there exists a $\mathcal U \subseteq \mathcal T$ such that $A = \bigcup \mathcal U$.
	
	\begin{proof}
		As $A$ is open, for any $x \in A$, there exists $\varepsilon _x\in \mathbb R_{> 0}$ such that $B(x, \varepsilon) \subseteq A$. Thus,
		$$
		\bigcup_{x \in A} B(x, \varepsilon_x) = A.
		$$
	\end{proof}
\end{corollary}



\section{Interior}
%--------------------------------------------


\begin{definition}
	\label{def: interior}
	
	Let $\mathbb X = (X, d)$ be a metric space.
	
	Given $A \subseteq X$, then the \textit{interior} of $A$, denoted by $A^\circ$, is defined to be the largest open set contained in $A$. Explicitly,
	$$
	A^\circ := \left\{ \bigcup \mathcal U: \mathcal U \in \mathcal T \cap 2^A \right\}.
	$$
\end{definition}


\begin{note}
	\label{note: interior: notation}
	
	Note that, the interior of $A$ is determined by $X$ and $d$ simultaneously. So, even if $\mathbb X_1 = (X, d_1)$ and $\mathbb X_2 = (X, d_2)$ are metric spaces with same set $X$, the interior of a subset $A \subseteq X$ might be different in these two spaces, if $d_1 \ne d_2$.
	
	For example, suppose $X = \mathbb R$, and let $d_1$ be a $L^p$ metric on $X$, and let $d_2$ be the discrete metric on $X$. Let $x \in X$, then, the interior of $\{x\}$ in $\mathbb X_1$ is empty, but in $\mathbb X_2$, the interior of $\{x\}$ is $\{x\}$ itself.
	
	In this case, for any $A \subseteq X$, we write $A^\circ_{d_1}$, $A^\circ_{\mathbb X_1}$, $\mathrm{Int}_{d_1}A$ or $\mathrm{Int}_{\mathbb X_1}A$ for the interior of $A$ in $\mathbb X_1$. Also, if $\mathcal T_1$ denoted for the topology on $\mathbb X_1$, we also write $A^\circ_{\mathcal T_1}$ or similar.
\end{note}


\begin{lemma}
	
\end{lemma}
























%